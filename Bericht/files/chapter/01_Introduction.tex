\chapter{Einführung}
Deadlocks sind häufig die Ursache, wenn ein Programm nicht mehr reagiert\cite{Joshi}.
Deadlocks entstehen, wenn sich mehrere Routinen in einem nebenläufigen 
Programm zyklisch blockieren. Locks, welche zu den häufigsten elementaren 
Methoden zur Synchronisierung in nebenläufigen Programmen gehöre, können 
leicht zu solchen Situationen führen \cite{zhou}. Dabei warten mehrere Routinen 
zyklisch auf die Freigabe eines oder mehrerer Locks, ohne dass die Möglichkeit
besteht, dass eines der Locks freigegeben wird. Eine weiter Situation, 
bei denen es zu Deadlocks kommen kann besteht darin, dass eine Routine 
ein Lock beansprucht, während es dasselbe Lock bereits hält.\\ 
Neben solchen Ressourcen-Deadlock gibt es noch weitere Möglichkeiten Deadlocks 
zu erzeugen, z.B. Kommunikations-Deadlocks, die im Folgenden aber nicht betrachtet 
werden sollen. \\
Da Deadlocks häufig nur in sehr bestimmten Fällen auftreten, kann es passieren,
dass solche Situationen während der Entwicklung eines Programms nicht bemerkt
werden. Dies führt z.B. dazu, dass in einer 2008 durchgeführten Stichprobe in 
vier open-source Anwendungen 
(MySql, Apache, Mozilla und OpenOffice) etwa $30 \%$ aller Concurrency-Probleme 
auf Deadlocks zurückzuführen waren \cite{Lu}.\\\\
Go besitzt eine sogenannte Go Runtime Deadlock Detection um Deadlocks zu erkennen. 
Dazu zählt Go die nicht blockierten Go-Routinen.
Fällt dieser Wert auf $0$, nimmt Go an, dass es zu einem Deadlock gekommen 
ist und bricht das Programm mit einer Fehlermeldung ab~\cite{grdd_code}.
Allerdings kann 
dies nur tatsächlich auftretende Deadlocks erkennen. Eine Erkennung, ob 
in dem Code ein Deadlock potenziell möglich ist findet hierbei nicht statt. Die 
Detektion erkennt auch nur Deadlocks, wenn sie alle Routinen betreffen. 
Wenn es Routinen gibt, die sich zyklisch blockieren, andere Routinen aber 
nicht blockiert sind, wird solch eine Situation nicht erkannt.\\\\
Da das Auftreten von Deadlocks sehr stark von dem tatsächlichen Ablauf eines 
Programms abhängt, kann ein Programm, bzw. eine Implementierung 
von Locks, welche potenzielle Deadlocks erkennen kann sehr hilfreich sein.
Die Betrachtung solcher Programme stellt den Hauptteil dieses Projekts da.
Dabei werden 
dynamische Programme betrachtet, also Programme, die während der Laufzeit des 
eigentlichen Programs nach Deadlocks suchen. Diese speichern Abhängigkeiten 
zwischen Lock-Operationen in Lock-Graphen oder Lock-Bäumen, und versuchen aus 
diesen auf Deadlocks zu schließen.\\\\
Der Bericht ist folgendermaßen aufgebaut: Zuerst wird auf den theoretischen 
Hintergrund bezüglich Locks und Deadlocks, sowie die Erkennung von 
potenziellen Deadlocks mit Lock-Graphen und Lock-Bäumen eingegangen. Außerdem 
wird die Funktionsweise des UNDEAD-Algorithmus \cite{zhou} betrachtet.
Anschließend
wird die Funktionsweise eines Detektors für Deadlocks ``go-deadlock'' 
\cite{sasha-s} analysiert, welcher auf Lock-Graphen basiert. Der Hauptteil 
des Projekts beschäftigte sich mit der Entwicklung und Implementierung eines 
auf Lock-Bäumen basierten Deadlock-Detektors für Go. Dieser basiert 
auf dem UNDEAD Algorithmus. Er wurde allerdings um weitere Funktionen,
wie z.B. RW-Locks erweitert. Dessen Aufbau und Funktionsweise wird in Kapitel 
\ref{Kap::Implementaion} beschrieben. Zum Schluss werden die beiden betrachteten 
Detektoren bezüglich ihrer Fähigkeit potenzielle Deadlocks bzw. 
Situationen, welche nicht zu Deadlocks führen können, korrekt zu erkennen, verglichen.
Außerdem 
wird der Einfluss der Detektoren auf die Laufzeit der Programme analysiert.


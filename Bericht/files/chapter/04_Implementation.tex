\chapter{Implementierung des Deadlock-Detektors \glqq Deadlock-Go \grqq} \label{Kap::Implementaion}
Im Folgenden soll die Funktionsweise und Implementierung eines selbst implementierten 
Detektors $\glqq$Deadlock-Go$\grqq$ 
betrachtet werden. Dieser wurde entwickelt um Ressourcen-Deadlocks in Go-Programmen
zu erkennen und den Nutzer vor solchen zu warnen. Der Code kann in 
\cite{implementation} gefunden werden.\\
Der Detektor basiert zum Teil auf dem UNDEAD-Algorithmus \cite{zhou}. 
Zusätzlich wurde er um RW-Locks
erweitert, welche in UNDEAD nicht betrachtet werden.\\
Der Detektor implementiert dabei Lock-Bäume, mit welchen zyklisches Locking 
erkannt werden kann. Neben den Informationen, die für die Erkennung solcher 
Zyklen benötigt werden, werden außerdem Information darüber gespeichert, wo in dem 
Programm-Code die Initialisierung sowie die (Try-)(R-)Lock Operationen aufgerufen werden. 
Diese werden dem Nutzer bereitgestellt, wenn ein potenzielles 
Deadlock erkannt wurde, um das Auffinden und Korrigieren des entsprechenden Deadlocks 
im Programmcode zu erleichtern.\\
Das folgende Programm ist ein Beispiel zur Verwendung des Detektors:
\begin{figure}[H]
  \lstinputlisting{code/implementation_ex_1.go}
\end{figure}
\newpage
Es erzeugt den folgenden Output:

\begin{mdframed}[leftmargin=10pt,rightmargin=10pt]
\textcolor{red}{POTENTIAL DEADLOCK}\\\\
\textcolor{purple}{Initialization of locks involved in potential deadlock:}\\\\
$/$home$/***/$undead$\_$test.go 4\\
$/$home$/***/$undead$\_$test.go 5\\\\
\textcolor{purple}{Calls of locks involved in potential deadlock:}\\\\
\textcolor{blue}{Calls for locks created at $/$home$/***/$undead$\_$test.go 4}\\
$/$home$/***/$undead$\_$test.go 21\\
$/$home$/***/$undead$\_$test.go 12\\\\
\textcolor{blue}{Calls for locks created at $/$home$/***/$undead$\_$test.go 5}\\
$/$home$/***/$undead$\_$test.go 20\\
$/$home$/***/$undead$\_$test.go 13\
\end{mdframed}

In diesem, sowie in allen folgenden Beispielen, wurden die Pfade durch $/***/$ 
gekürzt. In der tatsächlichen
Ausgabe wird der vollständige Pfad angegeben.\\ 
Der genaue Output hängt von dem tatsächlichen Ablaufs ab, der aufgrund der 
Nebenläufigkeit bei verschiedenen Durchläufen unterschiedlich sein kann.\\\\
Das restliche Kapitel ist in die folgenden Abschnitte eingeteilt:
\begin{itemize}
  \item Aufbau der Datenstrukturen
  \item (R-)Lock, (R-)TryLock und (R-)Unlock
  \item Periodische Detection
  \item Abschließende Detektion
  \item Meldung gefundener Deadlocks
  \item Optionen
\end{itemize}
\section{Aufbau der Datenstrukturen} \label{Kap::Implementation:Datastructures}
Im Folgenden sollen die in dem Detektor verwendeten Datenstrukturen betrachtet 
werden. Wie diese genau verwendet werde, wird in späteren Abschnitten noch 
genauer betrachtet.
\subsection{(RW-)Mutex}
Die Strukturen für die (RW-)Mutexe sind so implementiert, dass sie als 
Drop-In-Replacements für die klassischen sync.(RW)Mutexe verwendet werden.
Dabei gibt es eine Struktur für Mutexe und eine für RW-Mutexe, die über ein 
zusätzliches Interface MutexInt zusammengefasst werden können.\\
Die Mutex-Strukturen beinhalten alle Informationen, die für diese benötigt werden.
Dabei handelt es sich um die folgenden Informationen:
\begin{itemize}[leftmargin=1.3em]
  \item \texttt{mu} (*sync.(RW)Mutex) ist das eigentliche Lock, welches für das tatsächliche Locking 
      verwendet wird. Für ein Mutex ist dies ein *sync.Mutex, für ein RW-Mutex
      ein *sync.RWMutex.
  \item \texttt{context} ([]callerInfo) ist die Liste der callerInfo des Mutex.
    CallerInfo sind die Informationen darüber wo in dem Programm-Code die 
    Initialisierung und die (Try-)(R-)Lock Operationen stattgefunden haben. 
    Dazu werden die Datei und Zeilennummer gespeichert. Je nachdem, wie das 
    Programm über die Optionen konfiguriert ist, kann auch ein vollständiger 
    Call-Stack für die entsprechenden Operationen gespeichert werden. 
  \item \texttt{in} (bool) ist ein Marker welcher speichert, ob der Mutex richtig 
      initialisiert wurde. Es ist nicht mögliche ein Mutex einfach durch 
      $var\ x\ Mutex$ oder $var\ x\ RWMutex$ zu erzeugen und sofort zu verwenden, 
      sondern die Mutex Variable muss über $x := NewLock()$ bzw. 
      $x := NewRWLock()$ initialisiert werden. Wird eine Lock-Operation o.ä. 
      auf einem\\(RW-)Mutex ausgeführt, ohne dass dieses entsprechend initialisiert
      wurde, wird das Programm mit einem Fehler abgebrochen.
  \item \texttt{numberLocked} (int) speichert, wie oft das Lock im 
      moment gleichzeitig beansprucht ist. Ist das Lock frei, ist dieser Wert $0$.
      Bei einem Mutex kann der Wert maximal 1 werden, wenn das Lock gerade von 
      einer Routine gehalten wird. Das selbe ist wahr für RW-Mutexe, wenn diese 
      über ein Writer-Lock gehalten werden. Werden diese allerdings von 
      Reader-Locks gehalten, kann das selbe Lock von mehreren Locks gehalten 
      werden.
  \item \texttt{isLockedRoutineIndex} (*map[int]int) speichert die Indices der Routinen, von 
    welchen das Lock momentan gehalten wird, sowie wie oft es gehalten wird. 
    Dies wird für die Erkennung von doppeltem Locking verwendet.
  \item \texttt{memoryPosition} (uintptr) speichert die Speicheradresse des 
    Mutex Objekts bei seiner Erzeugung.
  \item \texttt{isRLock} (map[int]bool) (nur für RW-Mutex) speicher, von welcher Routine 
    der Mutex als RLock gehalten wird.
\end{itemize}
Zusätzlich enthalten die (RW)-Locks noch Locks, die bei der gleichzeitigen Verwendung 
des Locks durch R-Lock eine gleichzeitiges schreiben in maps verhindern.   
\subsection{Dependencies}
Dependencies werden verwendet, um die Abhängigkeiten der verschiedenen Mutexe 
untereinander zu speichern. Sie entsprechen dabei einer Menge an Kanten in einem 
Lock Graphen. Sie enthalten die folgenden Informationen:
\begin{itemize}
  \item \texttt{mu} (mutexInt) ist das Lock, für welches gespeichert werden soll, von 
   welchen anderen Locks mu abhängt, welche Locks also in der selben Routine 
   bereits gehalten wurden, als mu erfolgreich beansprucht wurde.
  \item \texttt{holdingSet} ([]mutexInt) ist die Liste der Locks, von denen mu abhängt.
  \item \texttt{holdingCount} (int) ist die Anzahl der Locks, von denen mu abhängt,
   also die Anzahl der Elemente in holdingSet.
\end{itemize}
\subsection{Routine}
Für jede Routine wird automatisch eine Struktur angelegt, die alle für sie 
relevanten Informationen speichert. Diese Strukturen werden in einem 
globalen Slice (Liste) routines gespeichert. Bei den Informationen, die von 
diesen Routinen gehalten werden handelt es 
sich vor allem um eine Liste der Mutexe, die momentan von der Routine 
gehalten werden, sowie die Inforationen, die für den Aufbau des Lock-Baums 
benötigt werden. Es handelt sich dabei um die folgenden Informationen:
\begin{itemize}
  \item \texttt{index} (int) ist der Index der Routine in routines.
  \item \texttt{holdingCount} (int) speichert die Anzahl der im Moment gehaltenen Locks.
  \item \texttt{holdingSet} ([]MutexInt) ist eine Liste, welche die
    momentan von der Routine gehaltenen Locks enthält.
  \item \texttt{dependencyMap} (map[uintptr]*[]*dependency) ist ein Dictionary, welches 
   verwendet wird, um zu verhindern, dass wenn die selbe Situation im Code 
   mehrfach auftritt (z.B. durch Schleifen) die entsprechenden Dependencies 
   mehrfach in dem Lock-Baum gespeichert werden.
  \item \texttt{dependencies} ([]*dependencies) speichert die aufgetretenen Dependencies.
   Dies stellt also die Implementierung des Lock-Baum da.
  \item \texttt{curDep} (*dependency) ist die letzte in den Lock-Baum (dependencies)
   eingefügte dependency.
  \item \texttt{depCount} (int) gibt die Anzahl der in dem Lock-Baum gespeicherten 
   Dependencies an.
  \item \texttt{collectSingleLevelLocks} speichert Informationen über single-level Locks, 
   also Locks, welche von keinem andere Lock abhängen. Auch dies ist wie 
   dependencyMap dazu da, um zu verhindern, dass Informationen, die
   bereits bekannt sind, mehrfach gespeichert werden. 
\end{itemize}
\section{(R-)Lock, (R-)TryLock, (R-)Unlock}
Im Folgenden sollen die Implementierungen der verschiedenen, auf den Mutexen
implementierten Operationen beschrieben werden.\\\\ 
Sind alle Teile des Detektors deaktiviert, werden nur die entsprechenden 
Operationen auf den sync.(RW)Mutex Locks ausgeführt. Eine weitere Sammlung von Daten
findet in diesem Fall nicht statt.\\\\
Ist der Detektor nicht vollständig deaktiviert, werden bei den verschiedenen 
Operationen Daten gesammelt, um die Detektion von Deadlocks zu ermöglichen.  
Bei dieser Sammlung werden die Datenstrukturen,
welche in \ref{Kap::Implementation:Datastructures} beschrieben wurden, verändert
oder neu erzeugt.\\
Diese Vorgänge sollen im Folgenden beschrieben werden.
\subsection{(R-)Lock}
(R)-Lock wird verwendet, um die Locks zu beanspruchen.\\
Das Locking von RW-Mutexes unterscheidet sich von dem Locking der Mutexe nur
dadurch, das isRLock in dem Mutex entsprechend gesetzt wird, und das 
Locking des eigentlichen sync.(RW)Locks entsprechend angepasst wird. Die 
Sammlung der Informationen, welche für die Detektion der Deadlocks benötigt
wird, ist die selbe.\\\\
Zuerst wird überprüft, ob das Lock sowie der Detektor bereits initialisiert 
wurde. Ist der Mutex nicht über $NewLock()$ initialisiert worden, wird das 
Programm mit einer Fehlermeldung abgebrochen. \\
Anschließend wird überprüft, ob für für die entsprechende Routine bereits ein 
Lock-Baum existiert, die Routine also bereits initialisiert worden ist. 
Ist dies nicht der Fall wird die Routine initialisiert und damit ein leere Baum 
für diese Routine erzeugt.\\
Solange die Detektion von doppeltem Locking nicht deaktiviert ist, wird nun
überprüft, ob das Beanspruchen dieses Locks zu einem Deadlock führen würde.
Dazu wird zuallererst überprüft, ob der Mutex momentan bereits von einer 
Routine gehalten wird. Ist dies nicht der Fall kann es nicht zu doppeltem Locking 
kommen. Wird es bereits gehalten, muss es dennoch nicht zu einem doppelten 
Locking kommen. Dies ist der Fall, wenn die Routine, die das Mutex hält, und 
die es momentan beansprucht nicht die selben sind oder wenn sie die selben 
sind und beide R-Locks sind. Ist nichts davon der Fall, nimmt das Programm an, 
es sei ein Deadlock gefunden worden. In diesem Fall wird eine Beschreibung des 
Deadlocks ausgegeben (vgl. \ref{Kap::Implementation:Report}). Anschließend wird 
die abschließende Detektion gestartet und das Programm anschließend abgebrochen.\\
Im Anschluss werden nun die entsprechenden Datenstrukturen aktualisiert. Dies geschieht 
allerdings nur, wenn momentan mindestens eine Routine läuft, da Deadlocks, abgesehen
von doppeltem Locking, nur bei der nebenläufigen Ausführung mehrerer Routinen 
auftreten. Man betrachte zu erst den Fall des Single-Level-Locks. Dabei handelt es 
sich um einen Mutex, der zu einem Zeitpunkt beansprucht wird an dem die 
selbe Routine keine anderen Locks hält. Da sich somit keine Dependencies bilden,
muss der Lock-Baum der Routine nicht verändert werden. In diesem Fall wird 
lediglich die Information über den Aufruf des Lockings in \texttt{context} gespeichert,
solange der Aufruf der exakt selben Beanspruchung (selbe Datei und selbe Zeilennummer)
noch nicht gespeichert wurde. Hält die Routine allerdings bereits ein oder 
mehrere Lock, wird die entsprechende Dependency in den Lock-Baum eingefügt,
solange sie in dieser noch nicht existiert. Diese Überprüfung wird mit Hilfe 
von \texttt{dependencyMap} ausgeführt. Dazu wird jeder Situation ein 
Key zugeordnet, die sich durch eine xOr Verknüpfung der Speicheradressen
des aktuellen und des zuletzt gelockten Locks in der Routine gebildet wird.
Existiert der Schlüssel nicht, kam die Situation noch nicht vor. Andernfalls 
wird überprüft, ob die momentane Situation tatsächlich einer bereits vorkommenden 
Situation entspricht, welche in der dependencyMap als mit dem Key verknüpften Value 
gespeichert werden.
Existiert die Situation noch nicht, wird sie in den 
Lock-Baum und in dependencyMap eingefügt. Für diese Dependency entspricht \texttt{mu}
gerade dem zu lockenden Mutex und \texttt{holdingSet} dem momentanen \texttt{holdingSet} 
der Routine, also der Liste aller Mutexe, die von der Routine im Moment gehalten 
werden.\\
Sowohl bei Single-Level-Locks als auch bei Locks, welche zu Dependencies führen
wird das Lock im Anschluss in das \texttt{holdingSet} der Routine eingefügt.\\\\
Sobald die Aktualisierung der Datenstrukturen abgeschlossen wurde wird die 
eigentliche Beanspruchung des Locks ausgeführt.

\subsection{(R)-TryLock}
Bei einer Try-Lock Operation wird ein Lock nur beansprucht, wenn es im Moment 
der Operation beansprucht werden kann, das Lock also nicht bereits von einer Routine 
gehalten wird. Aus diesem Grund kann die Beanspruchung des Locks nicht direkt 
zu einem Deadlock führen. Es ist also nicht notwendig nach doppeltem Locking 
zu suchen oder den Lock-Baum zu aktualisieren, wenn das Lock beansprucht wird.
Ein solches Locking kann nur zu einem Deadlock führen, wenn es bereits durch 
die (R)-TryLock-Operation gehalten wird und von einer anderen Operation ebenfalls 
beansprucht werden soll. Es wird also zuerst versucht das Lock zu beanspruchen.
Wenn die Beanspruchung erfolgreich war, wird angepasst, wie oft das Mutex gelockt ist,
das Mutex in das \texttt{holdingSet} der Routine eingefügt und anschließend zurück gegeben, ob die (R)-TryLock-Operation erfolgreich war
\subsection{(R)-Unlock}
Zuerst wird überprüft, ob der Mutex überhaupt gelockt ist. Ist dies nicht der 
Fall wird das Programm mit einer Fehlermeldung abgebrochen. Andernfalls
wird die Anzahl der Lockungen des Locks angepasst, dass Mutex aus dem 
\texttt{holdingSet} der Routine entfernt und das eigentliche Lock wieder freigegeben.

\section{Periodische Detektion} \label{Kap::Implementation:Periodical}
Ist sie nicht deaktiviert, so wird die periodische Detektion in regelmäßigen 
Abständen (default: 2s) gestartet
um nach lokalen, tatsächlich auftretenden Deadlocks zu suchen. Lokal bedeutet 
dabei, dass sich nur ein Teil der Routinen in einem Deadlock befindet. Sollte 
es zu einem totalen Deadlock kommen, bei dem alle Routinen blockiert werden, 
wird das Programm automatisch von der Go Runtime Deadlock Detection
beendet. In 
diesem Fall ist keine weitere abschließende Detektion von Deadlocks möglich.\\\\
Für die periodische Detektion wird von jeder Routine nur \texttt{curDep} betrachtet,
also diejenige Dependency, welche als letztes in den Lock-Baum eingefügt wurde. 
Die Detektion wird nur ausgeführt, wenn sich diese Menge seit der letzten 
periodischen Detektion verändert hat und momentan mindestens zwei Routinen im 
Moment ein Lock halten.
In diesem Fall wird versucht Zyklen in der Menge der \texttt{curDep} zu finden.\\
Dazu wird eine Depth-First-Search auf diesen Dependencies ausgeführt. Dazu wird 
zuerst die \texttt{curDep} einer der Routinen auf einen Stack gelegt. Der Stack 
entspricht immer dem momentan betrachteten Pfad. Anschließend wird für 
die \texttt{curDep} jeder Routine, die noch nicht auf dem Stack liegt und noch 
nicht zuvor bereits betrachtet wurde überprüft,
ob das Hinzufügen der Dependency zu einer gültigen Kette führt, ob also 
die Formeln \eqref{For::Theo:LockTree.a} - \eqref{For::Theo:LockTree.d} immer noch 
gelten. Ist dies der Fall, so wird überprüft, ob die Kette einen Kreis 
bildet, also ob die Formeln \eqref{For::Implementation:Periodical.e} und 
\eqref{For::Implementation:Periodical.f} ebenfalls gelten. Ist der Pfad mit der 
neuen Dependency eine gültige Kette aber kein Kreis, so wird die Dependency
auf den Stack gelegt und das ganze rekursiv wiederholt. Ist die Kette ein Kreis,
so nimmt dass Programm vorerst an, es sei ein lokaler Deadlock erkannt worden.
In diesem Fall wird überprüft, ob sich die HoldingSets der Routinen, von denen
sich eine Dependency in der Kette befindet, seit dem Begin der momentanen 
periodische Detektion verändert hat. Ist dies der Fall, so geht der Detektor
 davon aus, dass es sich um einen falschen Alarm handelt. Andernfalls 
wird dem Nutzer mitgeteilt, dass ein Deadlock gefunden wurde, es wird die 
abschließende Detektion gestartet und das Programm anschließend abgebrochen.
Gibt es keine Dependency die, wenn sie auf den Stack gelegt wird, zu einer 
gültigen Kette führt, so wird die oberste Dependency von dem Stack entfernt,
sodass andere mögliche Pfade betrachtet werden können.\\
Wenn es keinen Pfad gibt, der eine gültige, zyklisches Kette bildet, so geht 
der Detektor davon aus, dass sich das Programm nicht in einem lokalen Deadlock
befindet.

\section{Abschließende Detektion}
Die abschließende Detektion wird am Ende des Programs durchgeführt um 
potenzielle Deadlock zu finden, auch wenn diese in dem Durchlauf nicht 
tatsächlich aufgetreten sind. Sie muss vom Nutzer manuell in seinem Code 
gestartet werden, nachdem die Ausführung des eigentlichen Programms abgeschlossen 
wurde.. Sie wird in diesem Fall nur ausgeführt, wenn sie nicht 
deaktiviert ist und
 die Anzahl der Routinen, die in dem Programm vorkamen,
sowie die Anzahl der einzigartigen Dependencies mindestens zwei ist.\\
Im Großen und Ganzen verläuft die Detektion 
identisch zu der periodischen Detektion (vgl. \ref{Kap::Implementation:Periodical}).
Allerdings werden nun nicht nur die zuletzt in die Lock-Bäume aufgenommenen 
Dependencies, sondern alle in den Bäumen vorkommenden Dependencies betrachtet. 
Dabei wird darauf geachtet, dass von jeder Routine maximal eine 
Dependency in der Kette vorkommen kann. Außerdem wird die Detektion nicht beim 
ersten Auftreten eines potenziellen Deadlocks beendet, sondern erst, wenn alle 
möglichen Pfade betrachtet wurde.


\section{Meldung gefundener Deadlocks}\label{Kap::Implementation:Report}
Wird ein tatsächlicher oder potenzieller Deadlock gefunden, wird dem Nutzer 
dies durch eine Nachricht über den standard error file descriptor (Stderr)
mitgeteilt.
\subsection{Doppeltes Locking}
Tritt ein Fall von doppeltem Locking auf, so wird dem Nutzer das dabei 
involvierte Lock, sowie seine Aufrufe mitgeteilt. Im Folgenden ist ein Beispiel
für solch eine Ausgabe gegeben:
\begin{mdframed}[leftmargin=10pt,rightmargin=10pt]
  \textcolor{red}{DEADLOCK (DOUBLE LOCKING)}\\\\
  \textcolor{purple}{Initialization of lock involved in deadlock:}\\\\
  $/$home$/***/$undead$\_$test.go 238\\\\
  \textcolor{purple}{Calls of lock involved in deadlock:}\\\\
  $/$home$/***/$undead$\_$test.go 239\\
  $/$home$/***/$undead$\_$test.go 240\\\\
\end{mdframed}
\subsection{Deadlocks}
Bei einem tatsächlichen oder potenziellen Deadlock werden die Locks, welche in 
dem Zyklus, welcher den Deadlock erzeugt vorkommen, mit den Positionen
ihrer Initialisierung und ihren (Try-)(R-)Lock-Operationen angegeben. Dazu 
werden die callerInfo der Mutexe in den 
Dependencies betrachtet, die in dem Stack, welcher einen Deadlock beschreibt 
vorkommen.\\
Dies führt z.B. zu folgender Ausgabe:
\begin{mdframed}
\textcolor{red}{POTENTIAL DEADLOCK}\\
\\
\textcolor{purple}{Initialization of locks involved in potential deadlock:}\\
\\
$/$home$/***/$undead\_test.go 40\\
$/$home$/***/$undead\_test.go 39\\
\\
\textcolor{purple}{Calls of locks involved in potential deadlock:}\\
\\
\textcolor{blue}{Calls for lock created at: $/$home$/***/$undead\_test.go:40}\\
$/$home$/***/$undead\_test.go 48\\
$/$home$/***/$undead\_test.go 60\\
\\
\textcolor{blue}{Calls for lock created at: $/$home$/***/$undead\_test.go:39}\\
$/$home$/***/$undead\_test.go 47\\
$/$home$/***/$undead\_test.go 61
\end{mdframed}
Es ist möglich, sich statt nur der Datei und Zeilennummer auch einen 
Call-Stack anzeigen zu lassen:
\begin{mdframed}
\textcolor{red}{POTENTIAL DEADLOCK}\\
\\
\textcolor{purple}{Initialization of locks involved in potential deadlock:}\\
\\
/home$/***/$deadlockGo.go 24\\
/home$/***/$deadlockGo.go 25\\
\\
\textcolor{purple}{CallStacks of Locks involved in potential deadlock:}\\
\\
\textcolor{blue}{CallStacks for lock created at: $/$home$/***/$deadlockGo.go:24}\\
\\
goroutine 21 [running]:\\
DeadlockExamples/selfWritten.DeadlockGoPotentialDeadlock.func2()\\
        $\phantom{111}/$home$/***/$deadlockGo.go:43 +0x59\\
created by DeadlockExamples$/$selfWritten.DeadlockGoPotentialDeadlock\\
        $\phantom{111}/$home$/***/$deadlockGo.go:41 +0x14a\\
\\
goroutine 20 [running]:\\
DeadlockExamples$/$selfWritten.DeadlockGoPotentialDeadlock.func1()\\
        $\phantom{111}/$home$/***/$deadlockGo.go:33 +0x7b\\
created by DeadlockExamples$/$selfWritten.DeadlockGoPotentialDeadlock\\
        $\phantom{111}/$home$/***/$deadlockGo.go:29 +0xdb\\
\\
\textcolor{blue}{CallStacks for lock created at: $/$home$/***/$deadlockGo.go:25}\\
\\
goroutine 21 [running]:\\
DeadlockExamples$/$selfWritten.DeadlockGoPotentialDeadlock.func2()\\
        $\phantom{111}/$home$/***/$deadlockGo.go:42 +0x45\\
created by DeadlockExamples$/$selfWritten.DeadlockGoPotentialDeadlock\\
        $\phantom{111}/$home$/***/$deadlockGo.go:41 +0x14a\\
\\
goroutine 20 [running]:\\
DeadlockExamples$/$selfWritten.DeadlockGoPotentialDeadlock.func1()\\
        $\phantom{111}/$home$/***/$deadlockGo.go:34 +0x8f\\
created by DeadlockExamples/selfWritten.DeadlockGoPotentialDeadlock\\
        $\phantom{111}/$home$/***/$deadlockGo.go:29 +0xdb
\end{mdframed}
\section{Optionen}
Die Funktionsweise des Detektors kann über verschiedenen Optionen gesteuert
werden, die vor der der erste Lock-Operation gesetzt werden müssen.
Dies beinhaltet die folgenden Möglichkeiten:
\begin{itemize}
  \item Aktivierung oder Deaktivierung des gesamten Detektors (Default: aktiviert)
  \item Aktivierung oder Deaktivierung der periodische Detektion (Default: aktiviert)
  \item Aktivierung oder Deaktivierung der abschließenden Detektion (Default: aktiviert)
  \item Festlegung der Zeit zwischen periodischen Detektionen (Default: 2s)
  \item Aktivierung oder Deaktivierung der Sammlung von Call-Stacks (Default: deaktiviert)
  \item Aktivierung oder Deaktivierung der Sammlung von Informationen über Single-Level-Locks (Default: aktiviert)
  \item Aktivierung oder Deaktivierung der Detektion von doppeltem Locking (Default: aktiviert)
  \item Festlegung der maximalen Anzahl von Dependencies pro Routine (Default: 4096)
  \item Festlegung der maximalen Anzahl von Mutexe von denen ein Mutex abhängen kann (Default: 128)
  \item Festlegung der maximalen Anzahl von Routinen (Default: 1024)
  \item Festlegung der maximalen Länge eines Call-Stacks in Bytes (Default 2048)
\end{itemize}

\newpage
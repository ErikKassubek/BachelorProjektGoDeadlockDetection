\chapter{Implementierung eines Deadlock-Detectors}\label{Chap::Implementation}
Im folgenden soll die Implementierung eines Deadlock-Detektors basierend auf
dem Undead Algorithmus \cite{zhou} beschrieben werden. Die eigentliche 
Implementierung in Go kann in \cite{implementation} gefunden werden.\\
Der Detektor ist, wie auch der in \ref{Chap::Review:go-deadlock} beschriebene 
Detektor, als Drop-In-Replacement ausgelegt. Das bedeutet, dass die 
Funktionen des Detektors direkt in den Programm-Code integriert werden. Ein 
Beispiel dazu ist das folgende Programm:
\begin{figure}[H]
    \lstinputlisting{code/implementation_ex_1.go}
\end{figure}
Locks, die in Go normalerweise über sync.Mutex erzeugt und gehandhabt werden, 
werden dabei durch eigens definierte Locks ersetzt. Auf diesen können dann Lock, 
TryLock und Unlock Operationen ausgeführt werden.\\
Im folgende sollen die verschiedenen Phasen 
\begin{itemize}
    \item Optionen
    \item Initialisierung
    \item Logging
    \item Periodische Detektion
    \item Vollständige Detektion
    \item Meldung
\end{itemize}
des Detektors beschrieben werde. 


\section{Optionen} \label{Chap::Implementation:Options}
Die Funktionsweise des Detektors kann durch verschiedene Optionen 
verändert werden. Diese beinhalten die Aktivierung und Deaktivierung der 
periodischen und vollständigen Detektion, die Festlegung der Abstände, in denen 
die periodische Detektion aktiviert wird, die Entscheidung ob vollständige 
Call-Stacks oder nur vereinfachte Informationen gesammelt werden sowie 
ob Informationen über Single-level-Locks gesammelt werden sollen.\\
Zudem können maximale Werte für die Anzahl der Dependencies und Routinen, 
die maximale Größe der gesammelten Call-Stacks sowie die maximale Tiefe von 
der Lock-Bäume bestimmt werden. Diese Werte haben besonders Einfluss auf den 
benötigten Speicherplatz. Sind sie allerdings zu klein Gewählt, wird das 
Programm mit einer entsprechenden Fehlermeldung abgebrochen.\\
Diese Werte müssen, wenn sie anders als die default Werte gesetzt werden sollen 
vor der Initialisierung des Detektor gesetzt werden.

\section{Initialisierung} \label{Chap::Implementation:Initialisierung}
Sowohl der eigentliche Detektor als auch jede neue go-Routine muss initialisiert 
werden.\\
Die Initialisierung, welche über ``deadlock.Initialize()'' gestartet wird, hat
lediglich die Aufgabe, die periodische Detektion zu starten und in regelmäßigen 
Abständen auszuführen. Dazu wird eine go-Routine erzeugt, die über einen Ticker
die periodische Detektion in festgelegten Abständen (default: 2s) started. Die
eigentliche Funktionsweise der periodischen Detektion wird in Kapitel 
\ref{Chap::Implementation:Periodic} genauer erklärt.\\
Jeder go-Routine, welche durch ``deadlock.NewRoutine()'' kreiert wird, wird ein
Struct zugeordnet, welches neben dem Index der Routine auch alle Informationen 
speichert, die für den Aufbau des Lock-Baums nötig sind. Eine genauere 
Beschreibung dieser Informationen befindet sich im Kapitel 
\ref{Chap::Implementation:Logging}. Diese Routinen werden in einem globalen 
Slice ``routines'' gespeichert. Zusätzlich wird noch ein Dictionary ``mapIndex'' 
angelegt. Dieses ordnet für jede Routine dem Index dieser Routine die Position
in ``routines'' zu. 
Diese Aufteilung macht es möglich auf eine Routine
ohne langes suchen zuzugreifen. Da jede Routine nur auf sein eigenes Element in 
``routines'' zugreift, ist es so möglich, auf mehrere Elemente in ``routines''
gleichzeitig zuzugreifen, ohne die gesamte Liste für andere Routinen blocken zu 
müssen. Dies wäre nicht möglich, wenn die Routinen direkt in einem Dictionary
gespeichert werden würden.



\section{Logging-Phase} \label{Chap::Implementation:Logging}
Der Detektor implementiert Drop-in-Replacements für Locks, auf denen Lock,
TryLock und Unlock Operationen definiert sind. 
Diese führen das eigentliche Locking aus, und 
speichern gleichzeitig Informationen, die für die Detektion von Deadlocks 
benötigt werden.\\
Ein Lock besteht aus einem ``sync.Mutex'' Objekt, mit dem das eigentliche
Locking im Hintergrund ausgeführt wird, sowie einer Liste von ``callerInfo''
Objekten. Mit diesen werden Informationen darüber gespeichert, wo die 
Locks erzeugt wurden, und wo Logging Operationen ausgeführt wurden. Diese 
Informationen werden verwendet wenn ein tatsächlich auftretendes 
oder potentielles Deadlock gefunden wird, um die in diesen Deadlocks involvierten 
Locks in der Fehlermeldung zu identifizieren. Neben der Information, ob es sich 
um eine Lock-Initialisierung oder einen Lock-Vorgang handelt, werden je nachdem
wie die Optionen (vgl. \ref{Chap::Implementation:Options}) gesetzt sind entweder 
ein vollständiger Call-Stack oder lediglich Informationen über die Datei und 
Zeile, in denen der Vorgang im Programmcode vorkommt gespeichert.\\
Im folgenden sollen die drei auf Locks definierten Operationen genauer betrachtet
werden.
\subsection{Lock}\label{Chap::Implementation:Logging.Lock}
Der Lock-Vorgang besteht aus zwei Schritten. Zum einen wird das tatsächlich
sync.Mutex-Lock beansprucht. Zum anderen wird die Datenstruktur, welche später 
für die Detektion von Deadlocks verwendet wird aktualisiert. Dies geschieht 
allerdings nur, wenn momentan mehr als eine Routine läuft. UNDEAD geht davon
aus, dass Deadlock hauptsächlich durch die Kombination von mehreren gleichzeitig
laufenden Routinen entsteht. Solange demnach nur eine Routine läuft kann es 
nicht zu einem solchen Deadlock kommen, und die entsprechenden Datenstrukturen 
werden nicht aktualisiert, um die Performance des Programmes zu verbessern.
Die einige Möglichkeit, bei der ein Deadlock auftreten kann, wenn insgesamt 
nur eine Routine läuft, besteht in doppeltem Locking, wenn also ein Lock 
von der selben Routine mehrfach beansprucht wird, ohne zwischendurch frei 
gegeben zu werden. Allerdings ist UNDEAD auch in Situationen, in dehnen 
mehrere Routinen laufen nicht in der Lage, solche Deadlocks zu erkennen (s. 
\ref{Chap::Implementation:Periodic}), 
auch wenn die Detektion nicht besonders kompliziert wäre.\\
Eine andere Situation, bei der dies zu einem False-Negativ führen könnte, ist 
folgende. Man betrachte eine Situation mit zwei Routinen R1 und R2, welche 
gleichzeitig starten. Man nehme nun an, dass in einem Durchlauf R2 bereits 
beendet wurde, während R1 noch läuft. In diesem Fall werden Lock-Operationen,
in R1 nicht mehr betrachtet. Wenn R2 in einem erneuten Durchlauf nun allerdings 
länger benötigt als in dem ersten Durchlauf, können diese zuvor ignorierten 
Locks dennoch zu einem Deadlock führen, der im ersten Durchlauf aber nicht
erkannt werden konnte.\\
Angenommen es laufen mehr als eine Routine und Deadlock-Detektion ist in den 
Optionen nicht vollständig deaktiviert, wird die Datenstruktur zur 
Deadlock-Detektion aktualisiert. Jede Routine verfügt dabei, wie bereits in 
\ref{Chap::Implementation:Initialisierung} beschrieben, über eine eigene 
Datenstruktur, über die die entsprechenden Lock-Bäume aufgebaut werden. 
Eine Dependency wird dabei über ein Objekt dargestellt, welches eine Referenz 
zu einem Lock l, sowie eine Liste  ``holdingSet'' aller Locks, welche von der 
Routine bereits 
gehalten wurden, als l erfolgreich beansprucht wurde.\\
Die Datenstruktur beinhaltet die folgenden Variablen:
\begin{itemize}
    \item index: Jeder Routine wird ein Index zugewiesen.
    \item holdingCount: Diese Variable entspricht der Anzahl der momentan von der 
        Routine gehaltenen Locks
    \item holdingSet: Dieses Array beinhaltet Referenzen zu allen von dieser 
        Routinen momentan gehaltenen Locks.
    \item depCount: Dieser Integer beinhaltet die Anzahl der in dem Lock-Baum 
        gehaltenen Dependencies, also die Anzahl der Kanten in dem Lock-Baum.
    \item dependencies: Über dieses Array wird der eigentliche Lock-Baum 
        gehalten. Er besteht aus einer Liste aller in dem Baum gespeicherten 
        Dependencies.
    \item dependencyMap: Dieses Dictionary wird verwendet, um möglichst 
        effizient abzuschätzen, ob die momentane Situation bereits aufgetreten 
        ist, z.B. wenn sich das Lock in einer Schleife befindet. Als Werte wird
        eine Liste mit Referenzen zu Dependencies gespeichert, die bereits 
        betrachtet worden sind.
    \item curDep: Diese Variable beinhaltet eine Referenz zu der letzten 
        eingefügten Dependency.
    \item collectedSingleLevelLocks: Dieses Dictionary beinhaltet callerInfo für 
        singel-level Locks.
\end{itemize}
Eine Aktualisierung der Datenstruktur ist nur notwendig, wenn ``holdingCount'' 
größer 0 ist, da es sonst keine neuen Dependencies geben kann, oder wenn die 
Detektion von Double-Locks aktiviert ist. Bei Double-Locks handelt es sich 
um Locks, die beansprucht werden, während sie von der selben Routine 
bereits gehalten werden. In diesem Fall wird, bevor die Aktualisierung der 
Datenstruktur ausgeführt wird überprüft, ob sich das Lock bereits in dem
``holdingSet'' der entsprechenden Routine befindet. Ist dies der Fall, so wird 
das Programm mit einer Fehlermeldung abgebrochen. Eine solche Detektion ist in 
UNDEAD eigentlich nicht 
vorgesehen, in dieser Implementierung aber möglich, auch wenn sie standardmäßig 
deaktiviert ist.\\ Ist eine 
Aktualisierung notwendig, so wird zuerst der Schlüssel für ``dependencyMap''
berechnet, der der aktuellen Situation entspricht. Der Schlüssel berechnet sich 
über eine xOr Operation zwischen der Speicherposition des momentanen Locks 
sowie des letzten in ``currentHolding'' eingefügten Locks. Existiert dieser 
Schlüssel bereits, wird betrachtet, ob eine Dependency mit dem selben Lock
existiert, bei dem das ``holdingSet'' der Dependency dem momentanen ``holdingSet''
der Routine entspricht. Wenn diese nicht übereinstimmen, oder der Schlüssel
nicht gefunden wurde, geht das Programm davon aus, dass die momentan auftretenden
Dependencies zum ersten mal auftreten. Es wird dementsprechend eine neue
Dependencies in ``dependencies'' eingefügt, deren Lock aus dem Lock, auf dem die
Lock-Operation ausgeführt wird entspricht, und deren ``holdingSet'' eine 
Kopie des ``holdingSet'' der Routine ist. Zudem wird die neue Dependencies
in die Liste der Dependencies der ``dependencyMap'' hinzugefügt, und ``curDep''
auf die neu erzeugte Dependency gesetzt. Ist dies der Fall, dann wird eine 
Variable isNew auf true gesetzt, um zu Signalisieren, dass diese Situation 
noch nicht bekannt ist.
Dadurch kann sicher gestellt werden, dass für dieses Lock nicht erneut 
ein Call-Stack bzw. Caller Informationen gespeichert werden.\\
Es ist möglich in den Optionen einzustellen, dass Informationen über Lock 
Vorgänge auf single-level Locks gesammelt werden. Da auf single-level Locks keine
Dependencies definiert werden, kann nicht über ``depMap'' entschieden werden, 
ob der Lock-Vorgang schon einmal betrachtet worden ist. Um zu verhindern, dass 
der entsprechende Call-Stacks bzw. Caller-Informationen mehrfach gespeichert 
werden, werden Caller-Informationen (Datei und Zeile in denen die Lock Operation 
ausgeführt wird) in einem Dictionary gespeichert. Bei einem solchen Aufruf wird
nun überprüft, ob das Lock an der entsprechenden Stelle bereits einmal gelockt 
wurde. Ist dies der Fall, so wird isNew auf true gesetzt. Andernfalls wird der 
neue Lock Vorgang in das Dictionary eingefügt.\\
Wenn isNew auf true gesetzt wurde und entweder der holdingCount größer 0 ist 
oder die Sammlung von Caller-Informationen für single-Level Locks aktiviert ist,
wird entweder der ganze CallStack oder lediglich die Datei und Zeilennummer der 
Lock Operation in der Liste der ``callerInfo'' Objekte in dem Lock eingefügt.
Anschließend wird das Lock in ``currentHolding'', also die Liste der momentan
gehaltenen Locks eingefügt und ``holdingCount'' entsprechend angepasst.


\subsection{TryLock}\label{Chap::Implementation:Logging.TryLock}
TryLock versuchen ein Lock zu beanspruchen. Wenn es von keinem anderen Lock 
gehalten wird, geben sie true zurück und beanspruchen das entsprechende Lock. 
Ist es nicht möglich, dann geben sie false zurück. Anders als Lock warted das 
Programm in diesem Fall allerdings nicht, bis das Lock freigegeben wurde, sondern
führt das Programm weiter aus, ohne das Lock beansprucht zu haben.\\ 
TryLocks werden für die Detektion nur betrachtet, wenn die Beanspruchung 
erfolgreich war. In diesem Fall wird das Lock lediglich in ``holdingSet'' 
eingefügt.
Es werden allerdings nicht betrachtet, welche Locks bereits von der 
entsprechenden Routine gehalten werden und somit Dependencies erzeugen 
würden.

\subsection{UnLock}\label{Chap::Implementation:Logging.UnLock}
In der Unlock Operation wird das Lock lediglich aus ``holdingSet'' entfernt,
und ``holdingCount'' entsprechend angepasst. Anschließend wird das entsprechende
sync.Mutex-Lock freigegeben.

\section{Periodische Detektion} \label{Chap::Implementation:Periodic}
Die Aufgabe der periodischen Detektion besteht darin, tatsächlich auftretende
Deadlocks zu erkennen, und das Programm in diesem Fall abzubrechen. Sie wird,
wie in \ref{Chap::Implementation:Initialisierung} bereits beschrieben in 
regelmäßigen zeitigen Abständen automatisch gestartet, solange sie nicht 
über die Optionen deaktiviert wird. Sie wird außerdem nur dann durchgeführt,
wenn mindestens 2 Routinen laufen.\\
Der periodische Detektor besitzt ein Array ``lastHolding'', in dem für jede 
Routine, das Letzte erfolgreich beanspruchte Lock gespeichert wird. Bevor 
die eigentliche Detektion beginnt, werden die Werte dieser Liste für jede 
Routine mit den momentanen Werten für das zuletzt erfolgreich beanspruchte 
Lock verglichen. Gibt es keine Abweichungen, so nimmt der Detektor an, dass sich 
die situation seit der letzten periodischen Detektion nicht verändert hat, und 
führt keine erneute Detektion aus. Gibt es hingegen Abweichungen, so wird 
``lastHolding'' aktualisiert und die eigentliche Detektion gestartet.\\
Diese läuft nicht auf den Dependencies der eigentlichen Lock-Bäumen ab, 
sondern nur auf den zuletzt in die 
Lock-Bäume eingefügten Dependencies ``curDep''. Diese sind zur Erkennung von
Deadlocks ausreichend, die Betrachtung ist aber effizienter, da ihre Anzahl in 
der Regel geringer ist, als die Menge aller Dependencies. In diesen ``curDep's''
wird nun nach Zyklen gesucht, welche auf ein Deadlock hinweisen würden.
Dazu wird ein Stack verwendet. Zuerst wird das ``curDep'' $c_1$ einer der 
Routine $r$
auf den Stack gelegt. Das ganze wird insgesamt für alle Routinen als 
Startroutine wiederholt. Im nächsten Schritt werden alle Routinen betrachtet,
deren Index größer ist, als der Index der Routine, dessen ``curDep'' $c_1$ an 
oberster
Position im Stapel liegt. Es wird nun überprüft, ob sich mit dem ``curDep''
 $c_2$ dieser 
Routinen eine Kette bilden lässt, also ob das Lock von $c_1$ in dem 
``holdingSet'' von $c_2$ vorkommt. Wird ein solches 
``curDep'' gefunden, wird überprüft, ob die gebildete Kette einen Zyklus bildet.
Dies ist genau dann der Fall, wenn das Lock von $c_2$ in dem ``holdingSet''
der ``curDep'' auftaucht, welche an unterster Position im Stack liegt. Ist 
die Kette kein Zyklus, dann wird $c_2$ auf den Stapel gelegt und der Suchvorgang
rekursiv wiederholt, nun aber mit $c_2$ als neuem obersten ``curDep''. Dies wird
solange fortgesetzt, bis entweder ein Zyklus gefunden wurde, oder alle Routinen 
betrachtet wurden. Dabei wird darauf geachtet, dass nur Routinen betrachtet 
werden, deren ``curDep'' noch nicht auf dem Stapel liegen. Wenn ein Zyklus 
gefunden wurde geht der Detektor davon aus, dass sich das Programm in ein
Deadlock befindet. Um die Wahrscheinlichkeit von False-Positives zu verringern
wird nun noch einmal für alle in dem Zyklus auftauchenden Routinen überprüft,
ob sich das letzte Lock in dem ``holdingSet'' der Routine verändert hat. Ist dies 
der Fall, dann geht der Detektor von einem falschen Alarm aus. Andernfalls 
wird eine Fehlermeldung ausgegeben, die auf den Deadlock hinweist und die 
gesammelten ``callerInfo's'' zu den entsprechenden Locks ausgibt. Anschließend
wird noch die vollständige Detektion, wie in \ref{Chap::Implementation:Complete}
beschrieben ausgeführt und das Programm anschließend abgebrochen. 


\section{Vollständige Detektion} \label{Chap::Implementation:Complete}
Die vollständige Detektion wird nach Beendigung des Programms ausgeführt. 
Die Aufgabe dieses Teils besteht darin, den Lock-Baum, der in der Logging-Phase 
aufgebaut wurde nach potentiellen Deadlocks zu durchsuchen. Da UNDEAD 
sowieso nicht in
der Lage ist Deadlocks zu erkennen, die in nur einer Routine passieren (z.B. 
double locking), wird der Detektor nur ausgeführt, wenn in dem Programm 
mindestens zwei Routinen aufgetaucht sind. Außerdem zählt der Detektor, wie 
viele einzigartige Dependencies es in der Menge aller Lock-Bäume ist. Wenn es 
weniger als zwei einzigartige Routinen gibt, ist es ausgeschlossen, das 
der Detektor dort ein potentielles Deadlock findet. In diesem Fall wird die 
Suche abgebrochen.\\
Die Suche nach potentiellen Deadlocks läuft nahezu identisch ab wie die 
periodische Detektion
in \ref{Chap::Implementation:Periodic}. Allerdings werden hierbei nicht nur 
die zuletzt eingefügten, sondern alle in dem Lock-Baum enthaltenen Dependencies
betrachtet. Außerdem wird die Suche nicht bei dem ersten gefundenen Deadlock 
abgebrochen, sondern fortgeführt bis alle möglichen Pfade betrachtet wurden.


\section{Meldung}
Wird ein tatsächlicher oder potentieller Deadlock gefunden, wird dem Nutzer 
eine entsprechende Nachricht ausgegeben. Beide Methoden nutzen einen Stack um 
nach Zyklen zu suchen. Wenn ein solcher Kreis auftritt, sind die Locks, der in 
dem Stack gerade gehaltenen Dependencies gerade die Locks, die in dem Deadlock
involviert sind. Da die Caller-Informationen bzw. Stack-Traces direkt in den 
Locks gespeichert werden, können diese Informationen nun ausgegeben werden,
um dem Nutzer die Erkennung der (potentiellen) Deadlocks zu ermöglichen. 
Dabei wird zuerst ein Liste der Initialisierungen der Locks ausgegeben. 
Anschließend werden nach Locks sortiert die gesammelten Informationen über die 
Lock-Beanspruchungen ausgegeben.
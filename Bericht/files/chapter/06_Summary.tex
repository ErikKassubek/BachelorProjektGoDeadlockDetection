\chapter{Zusammenfassung}
Das Projekt betrachtete Möglichkeiten zur Detektion von Ressourcen-Deadlocks
in Go und entwickelte, sowie implementierte einen eigenen Detektor. Neben dem 
selbst implementierten Detektor Deadlock-Go wurde ein weiterer Detektor go-deadlock
betrachtet. Beide Detektoren benutzen dynamische Detektion um Deadlocks, welche 
durch Locks und RW-Locks erzeugt werden zu erkennen. Dabei nutzt go-deadlock 
Lock-Graphen, während Deadlock-Go auf Lock-Bäumen arbeitet. Beide Programme sind 
in der Lage viele solche Deadlocks zu erkennen, wobei Deadlock-Go in den 
Betrachteten Standartsituationen besser abschneidet, gerade auch was die
Vermeidung von Fasle-Positives bei RW-Locks angeht. Beide Detektoren 
haben einen negativen Einfluss auf die Laufzeit der Programme, wobei der 
genau Einfluss von der Anzahl der Locks und Routinen abhängt.
Nachdem die Detektoren in der Softwareentwicklung verwendet wurden, um potenzielle 
Deadlocks zu verhindern, können sie beide deaktiviert werden. Dadurch ist es 
möglich, dass die Programme nahezu ohne Verschlechterung der Laufzeit laufen 
können, ohne dass das Programm umgeschrieben werden muss.
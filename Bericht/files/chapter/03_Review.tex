\chapter{Review bestehender Tools}
\section{go-deadlock}\label{Chap::Review:go-deadlock}
Im folgenden soll eine Software zur Erkennung von Deadlocks analysiert werden.
Dazu wird die Software ``sasha-s/go-deadlock`` \cite{sasha-s}, veröffentlicht 
auf Github, betrachtet.\\
Dieses verwendet Drop-In Replacements für die, in go standardmäßig 
implementierten, sync.Mutex Locks. Diese führen sowohl das eigentliche 
Locking aus und können, beim Durchlaufen des Programms Situationen erkennen, 
die zu einem Deadlock führen können. Dabei werden sowohl ein allgemeines Lock 
als auch Readers–writer-locks implementiert. Dies wirkt sich allerdings nur auf
die Anwendung des eigentlichen Lockings aus, nicht aber auf die Detection von 
Deadlocks. Aus diesem Grund wird hierauf im folgenden nicht weiter eingegangen, 
und die Methoden für die Detection von Deadlocks bezieht sich sowohl auf die
allgemeinen als auch die Readers–writer Locks.\\
Für die Erkennung werden drei verschiedene Fälle betrachtet:
\begin{itemize}
    \item Rekursives Locking
    \item Zyklisches Locking
    \item Timeout
\end{itemize}
Diese sollen im folgenden genauer betrachtet werden.

\subsection{Rekursives Locking}\label{Kap::Rev:Recursive}
Unter rekursiven Locking versteht man, dass ein Lock von dem selben Thread 
mehrfach geschlossen wird, ohne zwischen diesen Schließungen wieder geöffnet zu
werden. Ein Beispiel dazu ist das folgende Programm:
\begin{figure}[H]
    \lstinputlisting{code/review_rekursive_1.go}
\end{figure}
Hierbei kann es schon, wie in dem Beispielcode gezeigt, bei nur einem Thread
zu einem Deadlock kommen. Die Funktion beansprucht das Lock $mu$ in Zeile $3$ 
für sich. In Zeile 4 versucht sie erneut das Lock $x$ zu beanspruchen. Da dieses
aber schon in Zeile 3 beansprucht wurde, muss das Programm vor Zeile 4 warten, 
bis $mu$ wieder freigegeben wird. Da dadurch aber die Freigabe in Zeile 3 
niemals erreicht werden kann wartet das Programm für immer, befindet sich also 
in einem Deadlock.\\
Um solche Situation zu erkennen speichert das Programm ein Dictionary $cur$ mit 
allen momentan gehaltenen 
Locks. Die Werte zu den jeweiligen Keys speichern sowohl, von welcher Go-Routine das 
Lock momentan gehalten wird, als auch in welcher Datei und Zeile der Befehl,
der das Locking dieses Locks zu folge hatte, zu finden ist. Wird ein Lock 
wieder freigegeben, so wird der entsprechende Eintrag aus $cur$ entfernt. 
Wird nun ein Lock neu beansprucht, überprüft 
das Programm, ob dieses Lock mit dem Thread, der das Lock beanspruchen möchte 
bereits in der Liste gehalten Locks auftaucht. Ist dies der Fall, dann nimmt das
Programm an, dass es sich hierbei um ein mögliches Deadlock handelt und führt 
entsprechende Schritte aus, um den Nutzer zu warnen. Wird ein Lock wieder frei
gegeben, wird der entsprechende Eintrag aus $l.loc$ entfernt.

\subsection{Zyklisches Locking}\label{Kap::Rev:Cyclic}
Mit dieser Methode werden mögliche Deadlocks gefunden, die dadurch entstehen,
dass alle Threads zyklisch auf die Freigabe eines Locks warten, welches von 
einem anderen Thread gehalten wird. Ein Beispiel dafür kann in folgender Funktion
gesehen werden:
\begin{figure}[H]
    \lstinputlisting{code/review_working_1.go}
\end{figure}
Dieses Beispiel entspricht dem in Kap.~\ref*{Kap::Theo:Deadlocks} beschreiben 
Beispiel.\\
Für die Detektion verwendet das Programm zwei Dictionaries. Das erste ist $cur$, 
welches bereits in \ref{Kap::Rev:Recursive} betrachten wurde.\\
Das andere Dictionary $order$, definiert mit seinen Keys die Kanten eines Lock-Graphen.
Die Keys bestehen dabei aus einer Struktur $beforeAfter$, die Referenzen zu den beiden
Locks speicherte, welche von der Kante im Graphen verbunden werden.
Wird ein neues Lock $p$ beanspruchen, so wird für 
jedes Lock $b$, welches sich momentan in $cur$ befindet ein  
neuer Eintrag $beforeAfter\{b, p\}$ in $order$ hinzugefügt. 
Die Werte, die für jeden Key in $order$ gespeichert werden, entsprechen 
den Information, die auch in $cur$ für die beiden Locks gespeichert wird. Allerdings 
wird auf die Speicherung der ID der erzeugenden Go-Routine verzichtet, da sie 
nicht benötigt wird.
Information aus $order$ werden nur entfernt, wenn $order$ eine festgelegte 
maximale Größe überschreitet. 
Die Überprüfung, ob ein Lock $p$ zu einem Deadlock führen kann, finden bereits statt,
bevor das Lock in $cur$ und $order$ eingetragen wird. Dazu wird für jedes Lock
$b$ in $cur$ überprüft,
ob $order$ einen Key $beforeAfter\{p, b\}$ besitzt, der Graph also die beiden 
Locks in 
umgekehrter Reihenfolge enthält. Existiert solch ein Key, und wurde $b$ 
von einer anderen Routine als $p$ in $cur$ eingefügt, bedeutet dies einen 
Loop aus zwei Kanten im Lockgraphen und somit einen möglichen Deadlock.\\
Dies bedeutet aber auch, dass das Programm nicht in der Lage ist, ein Kreis 
in einem Lock-Graphen zu finden, wenn dieser aus drei oder mehr Kanten besteht.
Soche Situationen können aber dennoch zu Deadlocks führen. Ein Beispiel dafür 
ist die folgende Funktion:
\begin{figure}[H]
    \lstinputlisting{code/review_3_edge_loop.go}
\end{figure}
Führen die drei Routinen jeweils ihre erste Zeile gleichzeitig aus, muss jede 
Routine vor ihrer zweiten Zeile warten und es kommt zu einem Deadlock. Da diese
Konstellation in einem Lockgraphen aber zu einem Kreis mit einer Länge von drei 
Kanten führen würde, kann das Programm den möglichen Deadlock nicht erkennen.\\
Ein weiteres Problem besteht darin, dass das Programm Einträge aus $l.order$ nicht 
löscht, wenn ein Lock wieder frei gegeben wird und auch nicht überprüft, das die 
Kanten in verschiedenen Go-Routinen erzeugt wurden. Sie werden nur entfernt, wenn 
das Dictionary eine bestimmte Größe überschreitet. Dies kann zu false-positives
führen. Ein Beispiel dazu wäre die folgende Funktion:  
\begin{figure}[H]
    \lstinputlisting{code/review_noDeletion.go}
\end{figure}
Es ist sehr einfach zu sehen, dass es in dieser Funktion nicht zu einem Deadlock 
kommen kann. Dennoch zeigt das Programm einen möglichen Deadlock an, da die 
Information über die Locks $x$ und $y$ aus Zeile $7-8$ in Zeile $10-11$ immer 
noch vorhanden ist, obwohl die beiden Locks in jedem Fall wieder freigegeben
werden, bevor Zeile $10$ erreicht wird.



\subsection{Timeout}
Neben diesen beiden Methoden, die vorausschauend nach möglichen Deadlocks 
Ausschau halten, versucht das Programm auch mit Timeouts um zu überprüfen ob 
sich das Programm bereits in einem Deadlock befindet. Möchte ein Thread ein 
Deadlock beanspruchen wird vorher eine go routine mit einem Counter gestartet.
Sollte die Inanspruchnahme des Locks innerhalb der vorgegebenen Zeit 
(default: 30s) gelingen, wird die go routine beendet. Sollte es nicht gelingen,
nimmt das Programm nach der festgelegten Zeit an, dass es zu einem Deadlock 
gekommen ist und gibt eine entsprechende Nachricht aus.\\
Diese Methode kann
durchaus nützlich sein, um über Deadlocks informiert zu werden. Allerdings führt
sie sehr leicht zu false-positives, wenn die Abarbeitung anderer Routinen und 
damit die Freigabe des Locks länger Dauer, als die festgelegte Timeout Zeit.
Im folgenden Beispiel wird dies deutlich:
\begin{figure}[H]
    \lstinputlisting{code/review_timeout_falsePositive.go}
\end{figure}
Der Chanel finished wird lediglich verwendet um zu verhindern, dass das Programm 
beendet wird, bevor die gol Routinen durchlaufen wurden. Er ist für die Deadlock
Analyse also irrelevant. Das Programm startet zwei go-Routinen, die beide das 
selbe Lock $x$ verwenden. Durch den time.Sleep(2 * time.Second) command wird 
sichergestellt, dass die erste go Routine zuerst auf das Lock zugreift. 
Dass Lock in Routine 2 muss also warten, bis es in Routine 1 wieder freigegeben
wird. Dies geschieht in etwa $38 s$ nachdem die zweite Routine mit dem warten 
auf die Freigabe von $x$ beginnt. Da dies länger ist als die standardmäßig
festgelegte maximale Wartezeit von $30 s$ nimmt das Programm an, es sei in einen 
Deadlock gekommen, obwohl kein Solcher vorliegt, und auch kein Deadlock möglich 
ist.

\subsection{False Negatives / Positives}
Bei False Negatives or False Positives handelt es sich um Fälle, bei denen es 
zu einem Deadlock kommen kann, ohne dass dieser Detektiert wird, bz. Fälle die
nicht zu einem 
Deadlock führen können, aber dennoch als Deadlock angezeigt werden. 
Neben den in den vorherigen Abschnitten gezeigten Fällen, kann es auch in weiteren 
Fällen zu solchem Verhalten kommen.\\
Ein Fall,
bei dem ein Deadlock auftreten könnte, welcher von dem Programm aber nicht erkannt
wird (False Negative), entsteht bei verschachtelten Routinen. Dabei erzeugt eine go-Routine eine 
weitere.
\begin{figure}[H]
    \lstinputlisting{code/review_nested.go}
\end{figure}
Diese Funktion ist bezüglich ihres Ablaufs identisch zu der Funktion 
circularLocking in Kapitel~\ref*{Kap::Rev:Cyclic}. Dennoch ist es dem 
Programm aufgrund der Verschachtelten go-Routine nicht möglich, den möglichen 
Deadlock zu erkennen. \\
Ein Fall, bei dem es zu False Positives kommen kann, wird durch Guard-Locks
ausgelöst. In dem folgenden Beispiel wird ein möglicher Deadlock der Locks $x$
und $y$ durch das Guard-Lock $z$ verhindert.
\begin{figure}[H]
    \lstinputlisting{code/review_guardLock.go}
\end{figure}
Da sich immer nur eine der beiden Routinen in dem Bereich aufhalten kann, der von 
$z$ begrenzt wird, ist ein Deadlock in diesem Fall nicht möglich. Es wird aber dennoch
ein möglicher Deadlock angezeigt, da das Programm nicht in der Lage ist, eine solches 
Guard-Lock zu erkennen.

\section{Go Runtime Deadlock Detection}
Go besitzt einen eigenen Detektor zur Erkennung von Deadlocks. Allerdings kann 
dieser, im Unterschied zu dem in~\ref{Chap::Review:go-deadlock} betrachteten 
Detector nur tatsächlich auftretende Deadlocks erkennen. Eine Erkennung ob 
in dem Code ein Deadlock möglich ist findet hierbei nicht statt. Um zu erkennen,
ob ein Deadlock vorliegt zählt go die nicht blockierten Go-Routinen.
Fällt dieser Wert auf $0$, nimmt Go an, dass es zu einem Deadlock gekommen 
ist und bricht das Programm mit einer Fehlermeldung ab~\cite{grdd_code}.

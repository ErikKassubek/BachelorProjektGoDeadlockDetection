%\pgfplotsset{compat=1.18}

% =Sprache=
%\usepackage[USenglish,british,american,australian]{babel}  %Für Englisch
\usepackage[ngerman]{babel} % Für deutsch
\usepackage[utf8]{inputenc} % Gibt utf8 also umlaute, ß etc
\usepackage{lmodern}    % update im der Latex schrift, macht schriftbild etwas schöner
\usepackage[T1]{fontenc}    % läd zeichensätze für westeuropäische sprachen nicht unbedingt nötig

% =Größe=
\usepackage[a4paper,lmargin={2.5cm},rmargin={2.5cm},tmargin={2.5cm},bmargin = {2.5cm}]{geometry}    % Erlaubt varriable einstellungen für abstände

% =Mathe=
\usepackage{amsmath}    % Grundlegendes packet, gibt Matheumgebungen
\usepackage{amssymb}    % Gibt Mathesymbole (pfeile etc) http://milde.users.sourceforge.net/LUCR/Math/mathpackages/amssymb-symbols.pdf
\usepackage{dsfont} % double struck fonts (z.B: für Mengen von Zahlen) http://milde.users.sourceforge.net/LUCR/Math/mathpackages/dsfont-symbols.pdf
\usepackage{mathrsfs}   % Geschwungene Buchstaben http://www.stat.colostate.edu/~vollmer/pdfs/typesetting-script.pdf
% es gibt da noch seeehr viele mehr...
\usepackage{nicefrac}
\usepackage{siunitx}
\usepackage{mathdots}

% =Bilder=
\usepackage{graphicx}   % Bilder einbinden
\usepackage{multirow}

\usepackage{pdfpages}
\usepackage{amsmath}

% =Malen=
%\usepackage[colorlinks]{hyperref}
\usepackage{tikz}   % Tool zum zeichnen von dingen
\usepackage{pgfplots, pgfplotstable}    % functions

% Pakete aus tikz, die geladen werden (gibt noch viel mehr).
\usetikzlibrary {positioning}
\usetikzlibrary{arrows,automata,shapes}
\usetikzlibrary{trees,fit,decorations.pathreplacing}    % group
\usetikzlibrary{calc} % calculate for relative positioning
\usetikzlibrary{graphs} %   graph placement
\usetikzlibrary{matrix,backgrounds} % Benötigt für hintergründe bei matrizen über tikz
% http://tex.stackexchange.com/questions/57152/how-to-draw-graphs-in-latex

\tikzset{near start abs/.style={xshift=1cm}}    % Einstellungen für shift


% =Optimierungen=
\graphicspath{{./Bilder/}}  % Gibt den standart ordner für bilder an
\usepackage{enumerate}  % gives extra options for \begin{enumerate}, e.g. use a),b) instead of 1),2)
\usepackage{hyperref}   % Klickbare Links
\usepackage{caption}    % Gives options for customising captions
\usepackage{subcaption} % different captions for multiple pictures in one float
%\usepackage{placeins}  % gibt \FloatBarrier. Wenn das benötigt wird, leben überdenken.
\usepackage{float,rotating} % Option for fix possitions; rotating of Graphics/tables
\usepackage{array,booktabs} % erweiterte Möglichkeiten zur Tabellendarstellung
\usepackage{longtable}  % gibt optionen für tabellen über mehrere Seiten
%\usepackage{uline} % allowes underlines, its strongly recomendet not to underline
%\setlength{\parindent}{0cm}    % Einrücktiefe auf 0cm setzen (siehe weiter unten im Dokument)
\usepackage{indentfirst}
\usepackage{enumitem}
\setlist{nolistsep}
\usepackage[justification=centering]{caption}

\newcommand{\MYhref}[3][blue]{\href{#2}{\color{#1}{#3}}}%
% \usepackage{titlesec}

\usepackage[utf8]{inputenc}
\usepackage[T1]{fontenc}
\usepackage{textcomp}
\usepackage{gensymb}


\usepackage{rotating}


\usepackage{makeidx}    % gives an Index
\makeindex  % Starts looking for index-keys
\newcommand{\Index}[1]{#1\index{#1}}    % gives the comand \Indext{test} which puts "`test"' in the Index, refering to the page the key is on.

% =Definitionen= (Siehe unten im Matheteil)
\newcommand{\pd}{\partial}
\newcommand{\pfrac}[2]{\frac{\partial #1}{\partial #2}}
%%%
% BODY
%%%
\setlength{\parindent}{0em}

\usepackage{fancyvrb}
\usepackage{fancyhdr}

\usepackage{amsthm}

\usepackage{tikz}

\usepackage{aligned-overset}

\usepackage[ruled,vlined,noend,linesnumbered]{algorithm2e}

\usepackage[backend=biber, style=ieee]{biblatex}
\bibliography{files/sources.bib}
\usepackage{csquotes}

% zu Zeigen
\newcommand{\zz}{$\mathrm{\raise0.25ex\hbox{Z}\kern-.56em\raise-0.25ex\hbox{Z}}$ }

\usepackage{pgfplots}
\pgfplotsset{compat=1.7}

\usepackage{graphicx}

\usepackage{listings}
\lstset{ % add your own preferences
    frame=single,
    basicstyle=\footnotesize,
    % keywordstyle=\color{red},
    numbers=left,
    numbersep=5pt,
    showstringspaces=false, 
    % stringstyle=\color{blue},
    tabsize=4,
    language=Go % this is it !
}